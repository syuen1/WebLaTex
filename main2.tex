\documentclass{llncs}
\usepackage{amsmath,amsfonts,amssymb}
\message{<Paul Taylor's Proof Trees, 2 August 1996>}
%% Build proof tree for Natural Deduction, Sequent Calculus, etc.
%% WITH SHORTENING OF PROOF RULES!
%% Paul Taylor, begun 10 Oct 1989
%% *** THIS IS ONLY A PRELIMINARY VERSION AND THINGS MAY CHANGE! ***
%%
%% 2 Aug 1996: fixed \mscount and \proofdotnumber
%%
%%      \prooftree
%%              hyp1            produces:
%%              hyp2
%%              hyp3            hyp1    hyp2    hyp3
%%      \justifies              -------------------- rulename
%%              concl                   concl
%%      \thickness=0.08em
%%      \shiftright 2em
%%      \using
%%              rulename
%%      \endprooftree
%%
%% where the hypotheses may be similar structures or just formulae.
%%
%% To get a vertical string of dots instead of the proof rule, do
%%
%%      \prooftree                      which produces:
%%              [hyp]
%%      \using                                  [hyp]
%%              name                              .
%%      \proofdotseparation=1.2ex                 .name
%%      \proofdotnumber=4                         .
%%      \leadsto                                  .
%%              concl                           concl
%%      \endprooftree
%%
%% Within a prooftree, \[ and \] may be used instead of \prooftree and
%% \endprooftree; this is not permitted at the outer level because it
%% conflicts with LaTeX. Also,
%%      \Justifies
%% produces a double line. In LaTeX you can use \begin{prooftree} and
%% \end{prootree} at the outer level (however this will not work for the inner
%% levels, but in any case why would you want to be so verbose?).
%%
%% All of of the keywords except \prooftree and \endprooftree are optional
%% and may appear in any order. They may also be combined in \newcommand's
%% eg "\def\Cut{\using\sf cut\thickness.08em\justifies}" with the abbreviation
%% "\prooftree hyp1 hyp2 \Cut \concl \endprooftree". This is recommended and
%% some standard abbreviations will be found at the end of this file.
%%
%% \thickness specifies the breadth of the rule in any units, although
%% font-relative units such as "ex" or "em" are preferable.
%% It may optionally be followed by "=".
%% \proofrulebreadth=.08em or \setlength\proofrulebreadth{.08em} may also be
%% used either in place of \thickness or globally; the default is 0.04em.
%% \proofdotseparation and \proofdotnumber control the size of the
%% string of dots
%%
%% If proof trees and formulae are mixed, some explicit spacing is needed,
%% but don't put anything to the left of the left-most (or the right of
%% the right-most) hypothesis, or put it in braces, because this will cause
%% the indentation to be lost.
%%
%% By default the conclusion is centered wrt the left-most and right-most
%% immediate hypotheses (not their proofs); \shiftright or \shiftleft moves
%% it relative to this position. (Not sure about this specification or how
%% it should affect spreading of proof tree.)
%
% global assignments to dimensions seem to have the effect of stretching
% diagrams horizontally.
%
%%==========================================================================

\def\introrule{{\cal I}}\def\elimrule{{\cal E}}%%
\def\andintro{\using{\land}\introrule\justifies}%%
\def\impelim{\using{\Rightarrow}\elimrule\justifies}%%
\def\allintro{\using{\forall}\introrule\justifies}%%
\def\allelim{\using{\forall}\elimrule\justifies}%%
\def\falseelim{\using{\bot}\elimrule\justifies}%%
\def\existsintro{\using{\exists}\introrule\justifies}%%

%% #1 is meant to be 1 or 2 for the first or second formula
\def\andelim#1{\using{\land}#1\elimrule\justifies}%%
\def\orintro#1{\using{\lor}#1\introrule\justifies}%%

%% #1 is meant to be a label corresponding to the discharged hypothesis/es
\def\impintro#1{\using{\Rightarrow}\introrule_{#1}\justifies}%%
\def\orelim#1{\using{\lor}\elimrule_{#1}\justifies}%%
\def\existselim#1{\using{\exists}\elimrule_{#1}\justifies}

%%==========================================================================

\newdimen\proofrulebreadth \proofrulebreadth=.05em
\newdimen\proofdotseparation \proofdotseparation=1.25ex
\newdimen\proofrulebaseline \proofrulebaseline=2ex
\newcount\proofdotnumber \proofdotnumber=3
\let\then\relax
\def\hfi{\hskip0pt plus.0001fil}
\mathchardef\squigto="3A3B
%
% flag where we are
\newif\ifinsideprooftree\insideprooftreefalse
\newif\ifonleftofproofrule\onleftofproofrulefalse
\newif\ifproofdots\proofdotsfalse
\newif\ifdoubleproof\doubleprooffalse
\let\wereinproofbit\relax
%
% dimensions and boxes of bits
\newdimen\shortenproofleft
\newdimen\shortenproofright
\newdimen\proofbelowshift
\newbox\proofabove
\newbox\proofbelow
\newbox\proofrulename
%
% miscellaneous commands for setting values
\def\shiftproofbelow{\let\next\relax\afterassignment\setshiftproofbelow\dimen0 }
\def\shiftproofbelowneg{\def\next{\multiply\dimen0 by-1 }%
\afterassignment\setshiftproofbelow\dimen0 }
\def\setshiftproofbelow{\next\proofbelowshift=\dimen0 }
\def\setproofrulebreadth{\proofrulebreadth}

%=============================================================================
\def\prooftree{% NESTED ZERO (\ifonleftofproofrule)
%
% first find out whether we're at the left-hand end of a proof rule
\ifnum  \lastpenalty=1
\then   \unpenalty
\else   \onleftofproofrulefalse
\fi
%
% some space on left (except if we're on left, and no infinity for outermost)
\ifonleftofproofrule
\else   \ifinsideprooftree
        \then   \hskip.5em plus1fil
        \fi
\fi
%
% begin our proof tree environment
\bgroup% NESTED ONE (\proofbelow, \proofrulename, \proofabove,
%               \shortenproofleft, \shortenproofright, \proofrulebreadth)
\setbox\proofbelow=\hbox{}\setbox\proofrulename=\hbox{}%
\let\justifies\proofover\let\leadsto\proofoverdots\let\Justifies\proofoverdbl
\let\using\proofusing\let\[\prooftree
\ifinsideprooftree\let\]\endprooftree\fi
\proofdotsfalse\doubleprooffalse
\let\thickness\setproofrulebreadth
\let\shiftright\shiftproofbelow \let\shift\shiftproofbelow
\let\shiftleft\shiftproofbelowneg
\let\ifwasinsideprooftree\ifinsideprooftree
\insideprooftreetrue
%
% now begin to set the top of the rule (definitions local to it)
\setbox\proofabove=\hbox\bgroup$\displaystyle % NESTED TWO
\let\wereinproofbit\prooftree
%
% these local variables will be copied out:
\shortenproofleft=0pt \shortenproofright=0pt \proofbelowshift=0pt
%
% flags to enable inner proof tree to detect if on left:
\onleftofproofruletrue\penalty1
}

%=============================================================================
% end whatever box and copy crucial values out of it
\def\eproofbit{% NESTED TWO
%
% various hacks applicable to hypothesis list 
\ifx    \wereinproofbit\prooftree
\then   \ifcase \lastpenalty
        \then   \shortenproofright=0pt  % 0: some other object, no indentation
        \or     \unpenalty\hfil         % 1: empty hypotheses, just glue
        \or     \unpenalty\unskip       % 2: just had a tree, remove glue
        \else   \shortenproofright=0pt  % eh?
        \fi
\fi
%
% pass out crucial values from scope
\global\dimen0=\shortenproofleft
\global\dimen1=\shortenproofright
\global\dimen2=\proofrulebreadth
\global\dimen3=\proofbelowshift
\global\dimen4=\proofdotseparation
\global\count255=\proofdotnumber
%
% end the box
$\egroup  % NESTED ONE
%
% restore the values
\shortenproofleft=\dimen0
\shortenproofright=\dimen1
\proofrulebreadth=\dimen2
\proofbelowshift=\dimen3
\proofdotseparation=\dimen4
\proofdotnumber=\count255
}

%=============================================================================
\def\proofover{% NESTED TWO
\eproofbit % NESTED ONE
\setbox\proofbelow=\hbox\bgroup % NESTED TWO
\let\wereinproofbit\proofover
$\displaystyle
}%
%
%=============================================================================
\def\proofoverdbl{% NESTED TWO
\eproofbit % NESTED ONE
\doubleprooftrue
\setbox\proofbelow=\hbox\bgroup % NESTED TWO
\let\wereinproofbit\proofoverdbl
$\displaystyle
}%
%
%=============================================================================
\def\proofoverdots{% NESTED TWO
\eproofbit % NESTED ONE
\proofdotstrue
\setbox\proofbelow=\hbox\bgroup % NESTED TWO
\let\wereinproofbit\proofoverdots
$\displaystyle
}%
%
%=============================================================================
\def\proofusing{% NESTED TWO
\eproofbit % NESTED ONE
\setbox\proofrulename=\hbox\bgroup % NESTED TWO
\let\wereinproofbit\proofusing
\kern0.3em$
}

%=============================================================================
\def\endprooftree{% NESTED TWO
\eproofbit % NESTED ONE
% \dimen0 =     length of proof rule
% \dimen1 =     indentation of conclusion wrt rule
% \dimen2 =     new \shortenproofleft, ie indentation of conclusion
% \dimen3 =     new \shortenproofright, ie
%                space on right of conclusion to end of tree
% \dimen4 =     space on right of conclusion below rule
  \dimen5 =0pt% spread of hypotheses
% \dimen6, \dimen7 = height & depth of rule
%
% length of rule needed by proof above
\dimen0=\wd\proofabove \advance\dimen0-\shortenproofleft
\advance\dimen0-\shortenproofright
%
% amount of spare space below
\dimen1=.5\dimen0 \advance\dimen1-.5\wd\proofbelow
\dimen4=\dimen1
\advance\dimen1\proofbelowshift \advance\dimen4-\proofbelowshift
%
% conclusion sticks out to left of immediate hypotheses
\ifdim  \dimen1<0pt
\then   \advance\shortenproofleft\dimen1
        \advance\dimen0-\dimen1
        \dimen1=0pt
%       now it sticks out to left of tree!
        \ifdim  \shortenproofleft<0pt
        \then   \setbox\proofabove=\hbox{%
                        \kern-\shortenproofleft\unhbox\proofabove}%
                \shortenproofleft=0pt
        \fi
\fi
%
% and to the right
\ifdim  \dimen4<0pt
\then   \advance\shortenproofright\dimen4
        \advance\dimen0-\dimen4
        \dimen4=0pt
\fi
%
% make sure enough space for label
\ifdim  \shortenproofright<\wd\proofrulename
\then   \shortenproofright=\wd\proofrulename
\fi
%
% calculate new indentations
\dimen2=\shortenproofleft \advance\dimen2 by\dimen1
\dimen3=\shortenproofright\advance\dimen3 by\dimen4
%
% make the rule or dots, with name attached
\ifproofdots
\then
        \dimen6=\shortenproofleft \advance\dimen6 .5\dimen0
        \setbox1=\vbox to\proofdotseparation{\vss\hbox{$\cdot$}\vss}%
        \setbox0=\hbox{%
                \advance\dimen6-.5\wd1
                \kern\dimen6
                $\vcenter to\proofdotnumber\proofdotseparation
                        {\leaders\box1\vfill}$%
                \unhbox\proofrulename}%
\else   \dimen6=\fontdimen22\the\textfont2 % height of maths axis
        \dimen7=\dimen6
        \advance\dimen6by.5\proofrulebreadth
        \advance\dimen7by-.5\proofrulebreadth
        \setbox0=\hbox{%
                \kern\shortenproofleft
                \ifdoubleproof
                \then   \hbox to\dimen0{%
                        $\mathsurround0pt\mathord=\mkern-6mu%
                        \cleaders\hbox{$\mkern-2mu=\mkern-2mu$}\hfill
                        \mkern-6mu\mathord=$}%
                \else   \vrule height\dimen6 depth-\dimen7 width\dimen0
                \fi
                \unhbox\proofrulename}%
        \ht0=\dimen6 \dp0=-\dimen7
\fi
%
% set up to centre outermost tree only
\let\doll\relax
\ifwasinsideprooftree
\then   \let\VBOX\vbox
\else   \ifmmode\else$\let\doll=$\fi
        \let\VBOX\vcenter
\fi
% this \vbox or \vcenter is the actual output:
\VBOX   {\baselineskip\proofrulebaseline \lineskip.2ex
        \expandafter\lineskiplimit\ifproofdots0ex\else-0.6ex\fi
        \hbox   spread\dimen5   {\hfi\unhbox\proofabove\hfi}%
        \hbox{\box0}%
        \hbox   {\kern\dimen2 \box\proofbelow}}\doll%
%
% pass new indentations out of scope
\global\dimen2=\dimen2
\global\dimen3=\dimen3
\egroup % NESTED ZERO
\ifonleftofproofrule
\then   \shortenproofleft=\dimen2
\fi
\shortenproofright=\dimen3
%
% some space on right and flag we've just made a tree
\onleftofproofrulefalse
\ifinsideprooftree
\then   \hskip.5em plus 1fil \penalty2
\fi
}

%==========================================================================
% IDEAS
% 1.    Specification of \shiftright and how to spread trees.
% 2.    Spacing command \m which causes 1em+1fil spacing, over-riding
%       exisiting space on sides of trees and not affecting the
%       detection of being on the left or right.
% 3.    Hack using \@currenvir to detect LaTeX environment; have to
%       use \aftergroup to pass \shortenproofleft/right out.
% 4.    (Pie in the sky) detect how much trees can be "tucked in"
% 5.    Discharged hypotheses (diagonal lines).

\newtheorem{Def}{Definition}
\newtheorem{Lem}{Lemma}
\newcommand{\nnegreal}{\mathbb{R}^{\geq 0}}
\newcommand{\ulim}[1]{\mathsf{ub}(#1)}
\newcommand{\llim}[1]{\mathsf{lb}(#1)}
\newcommand{\zsub}[1]{(\!\vert{#1}\vert\!)}

\title{A Zone-based Reachability Analysis for Nested Timed Automata}
\author{Seiichirou Tachi\inst{1} \and Shoji Yuen\inst{1} \and Mizuhito Ogawa\inst{2}}
\institute{Graduate School of Informatics, Nagoya University, Japan \and
Japan Advanced Institute of Science and Technology< Japan}
\begin{document}
\maketitle

\section{Introduction}

\section{Timed Automata}

$\nnegreal$ denotes the set of non-negative real numbers.  An {\em interval}
is one of $[\ell,u]$, $(\ell,u]$, $[\ell,u')$, and $(\ell, u')$ where
$\ell,u\in\nnegreal$ and $u'\in\nnegreal\cup\{\infty\}$.  For an
interval, `$[$' and `$]$' are the {\em closed} lower bound and upper bound
respectively and `$($' and `$)$' are the {\em open} lower bound and upper bound.
The open upper bound $\infty$ shows the interval has no upper bound.  The
set of all intervals is $\mathcal{I}$ For an interval $I$, $\llim{I}=\ell$,
$\ulim{I}=u'$ or $\ulim{I}=u$ for
$[\ell,u]$, $(\ell,u)$, $[\ell,u')$, and $(\ell,u')$.

\begin{Def}[Timed Automata]
  A timed automaton is a tuple $\mathcal{A}=(Q,q_0,X,\Delta)$ where
  \begin{itemize}
    \item $Q$ is a finite set of control locations 
    with the initial location $q_0\in Q$,
  %  \item $F\subseteq Q$ is the set of final locations,
    \item $X$ is a finite set of clocks,
    \item $\Delta\subseteq Q\times\mathcal{O}\times Q$ where $\mathcal{O}$ is a set of operations.
    A transition $\delta\in\Delta$ is a triplet $(q_1,\phi,q_2)$ written as $q_1\xrightarrow{\phi}q_2$
    with $\phi\in\mathcal{O}$ as one of the following operations over $X$:
    \begin{list}{}{}
      \item [{\bf Local}] $\epsilon$, an internal operation;
      \item [{\bf Test}] $x\in I?$ where $x$ is a clock and $I\in\mathcal{I}$; and
      \item [{\bf Assignment}] $x\leftarrow I$ where $x$ is a clock and $I\in\mathcal{I}$.
    \end{list}
  \end{itemize}
\end{Def}

We write $\mathcal{A}(X)$ to explicitly show that $\mathcal{A}$'s clocks are $X$.

A clock valuation over $X$ is a map from $X$ to $\nnegreal$.   The set of clock valuations over $X$
is written as $N_X$.  The clock valuation assigning 0 to $X$ is denoted by $0_X$.

\begin{Def}[Semantics of Timed automata]
  Given a timed automaton $\mathcal{A}=(Q,q_0,X,\Delta)$, a configuration
  of $\mathcal{A}$ is $(q,\nu)$ where $q\in Q$ and $\nu\in N_X$.
  A transition $(q_1,\nu_1)\xrightarrow{\phi}(q_2,\nu_2)$ where
  \begin{itemize}
    \item $(q_1,\nu_1)\xrightarrow{\epsilon}(q_2,\nu_2)$ with $\nu_1=\nu_2$ and $(q_1,\epsilon,q_2)\in\Delta$;
    \item $(q_1,\nu_1)\xrightarrow{x\in I?}(q_2,\nu_2)$ with $(q_1,x\in I?,q_2)\in\Delta$, $\nu_1=\nu_2$ and $\nu_1(x)\in I$; and
    \item $(q_1,\nu_1)\xrightarrow{x\leftarrow I}(q_2,\nu_2)$ with $(q_1,x\leftarrow I,q_2)\in\Delta$, $\nu_2=\nu_1[x\mapsto r]$ where $r\in I$
  \end{itemize}
\end{Def}

\section{Nested Timed Automata}

\begin{Def}
  A nested timed automaton (NeTA) is given by $\mathcal{N}=(T,\mathcal{A}_0,\Delta,X_g)$ where
  \begin{itemize}
    \item $T=\{\mathcal{A}_0,\cdots,\mathcal{A}_n\}$ is a set of timed automata whose clocks are $X_c$ with
    the transitions $(q_1,\phi,q_2)\in\Delta(\mathcal{A}_i)$ where $\phi$ is an operation over $X_c\cup X_g$,
    \item $A_0\in T$ is the initial timed automaton,
    \item $X_g$ is the set of global clocks, and
    \item $\Delta_g\subseteq (T\times\{\mathsf{push}\}\times TT)\cup(TT(Q(T))\times\{\mathsf{pop}\}\times T(Q(T)))$ with
    \begin{list}{}{}
      \item[$\mathsf{push}$] $\mathcal{A}_i\xrightarrow{\mathsf{push}}\mathcal{A}_i\mathcal{A}_j$, and
      \item[$\mathsf{pop}$] $\mathcal{A}_i\mathcal{A}_j(q_1)\xrightarrow{\mathsf{pop}}\mathcal{A}_i(q_2)$ with 
      $q_1\in Q(\mathcal{A}_j),q_2\in Q(\mathcal{A_i})$
    \end{list}
  \end{itemize}
  $Q(\mathcal{A}_i)$ for $\mathcal{A}_i\in T$ is assumed mutually disjoint.  We write $q(\mathcal{A}_i)$ for $q$ to show $q\in Q(\mathcal{A}_i)$. 
\end{Def}

\begin{Def}
  A configuration of NeTA is given by $(c,\nu_g)$ where $c\in\mathcal{C}^\ast$ with 
  $\mathcal{C}=\bigcup_i Q(\mathcal{A}_i)\times \nu_{X_c}$ and $\nu_g\in \nu_{X_g}$.

  \begin{itemize}
    \item $(c_1,\nu_g^1)\xrightarrow{t}(c_2,\nu_g^2)$ where $t\in\mathbb{R}_{\geq 0}$, $c_2=c_1+t$ and $\nu_g^2=\nu_g^1+t$;
    \item $(c(q_1(\mathcal{A}_i),\nu_c^1),\nu_g^1)\xrightarrow{\phi}(c(q_2(\mathcal{A}_i),\nu_c^2),\nu_g^2)$ where 
    $(q_1,\nu_c^1\cup \nu_g^1)\xrightarrow{\phi}(q_2,\nu_c^1\cup \nu_g^2)$ is a transition of $\mathcal{A}_i(X_c\cup X_g)$;
    \item $(c(q_1(\mathcal{A}_i),\nu_c^1),\nu_g)\xrightarrow{\mathsf{push}}(c(q_1(\mathcal{A}_i),\nu_c^1)(q_0(\mathcal{A}_j),0_{X_c}),\nu_g)$
    where $q_0$ is the initial location of $\mathcal{A}_j$ when $\mathcal{A}_i\xrightarrow{\mathsf{push}}\mathcal{A_i}\mathcal{A_j}\in\Delta_g$; and
    \item $(c(q_1(\mathcal{A}_i),\nu_c^1)(q(\mathcal{A}_j),\nu_c^2),\nu_g)\xrightarrow{\mathsf{pop}}(c(q_2(\mathcal{A}_i),\nu_c^1),\nu_g)$
    when $\mathcal{A}_i(q_1)\mathcal{A}_j\xrightarrow{\mathsf{pop}}\mathcal{A}_i(q_2)\in\Delta_g$
  \end{itemize}
\end{Def}

\section{Push-down automata over Zones}

For a set of items $Y$,
A {\em zone} over $Y$ is a set of clock valuations given by a conjunction of constraints 
of the form $x-y\triangleleft c$ where $x,y\in Y$, $c\in\mathbb{Z}$, and
$\triangleleft\in\{<,\leq\}$.

For a zone $Z$, $DBM(Z)$ is a set of quadruples $(x,y,c,\triangleleft)$ where $x,y\in Y$, $c\in\mathbb{Z}$, and 
$\triangleleft\in\{<,\leq\}$.   Each quadruple is written as $x-y\triangleleft c$.

For clocks $X$, the PDZ-items $Z_x=X\cup X^\bullet\cup\{\vdash,\vdash^\bullet



\begin{itemize}
  \item $\mathsf{Test}(Z,x\in I)=Z\wedge\{x-\mathbf{0}\triangleleft\ulim{I},\mathbf{0}-x\triangleleft-\llim{I}\}$;
  \item $\mathsf{Free}(Z,Y)=(Z\ominus Y)\oplus Y$;
  \item $\mathsf{Set}(Z,x\leftarrow I)=\mathsf{Free}(Z,\{x\})\wedge\{x-\mathbf{0}\triangleleft\ulim{I},\mathbf{0}-x\triangleleft-\llim{I}\}$;
  \item $\mathsf{Reset}(Z,Y)=\mathsf{Free}(Z,Y)\wedge\{y-\mathbf{0}\leq 0\;|\;y\in Y\}$;
  \item $\mathsf{Copy}(Z,x\leftarrow y)=\mathsf{Free}(Z,\{x\})\wedge\{x-y\leq 0,y-x\leq 0\}$;
  \item $Z\zsub{y\mapsto z}=\mathsf{Copy}(Z,z\leftarrow y)\ominus\{y\}$
\end{itemize}

\section{Simulation}

A binary relation $\preceq$ on configurations of timed automata $(q,v)$ is
a {\em simulation} if, whenenver $(q,v)\preceq (q',v')$, we have
$q=q'$ and
\begin{itemize}
  \item for each delay $t\in\mathbb{R}_{\leq 0}$, $(q,v+t)\preceq (q,v'+t)$;
  \item $(q,v)\xrightarrow{\epsilon}(q_1,v)$ implies 
  that $(q,v')\xrightarrow{\epsilon}(q_1,v')$;
  \item $(q,v)\xrightarrow{x\in I?}(q_1,v)$ implies
  $(q,v')\xrightarrow{x\in I?}(q_1,v)$; and
  \item $(q,v)\xrightarrow{x\leftarrow I}(q_1,\nu_1)$ implies there exists $v'_1$
  such that $(q,v')\xrightarrow{x\leftarrow I}(q_1,v'_1)$ and 
  $(q_1,\nu_1)\preceq (q_1,v'_1)$
\end{itemize}

We write $v\preceq_q v'$ for $(q,v)\preceq (q,v')$.  Given a pair of zones $Z,Z'$,
$Z\preceq_q Z'$ if for all $v\in Z$ there exists $v'\in Z'$ such that
$v\preceq_q v'$.

\begin{Lem}
  \begin{itemize}
    \item If $(q,Z)\xrightarrow{\phi}(q_1,Z_1)$ and $Z\preceq_q Z'$, then
  for some $Z'_1$, $(q,Z')\xrightarrow{\phi}(q_1,Z'_1)$ and $Z_1\preceq Z'_1$; and
    \item If $(q,Z)\xrightarrow{t}(q,Z_1)$ and $Z\preceq_q Z'$, then 
  there exists $Z'_1$ such that $(q,Z')\xrightarrow{t}(q,Z'_1)$ and 
  $Z_1\preceq Z'_1$
  \end{itemize}
\end{Lem}

\[
\begin{prooftree}
{}\justifies
{\mathfrak{S}:=\{(q_0(A_0),Z_0)\}, \mathcal{S}_{(q_0(\mathcal{A}_0),Z_0)}:=\{(q_0(\mathcal{A}_0),Z_0)\}}
\using{[\mathsf{start}]}
\end{prooftree}
\]

\[
  \begin{prooftree}
    {(q,Z)\in\mathfrak{S}\quad (q',Z')\in\mathcal{S}_{q,Z}\quad q'\xrightarrow{\epsilon}q''}
    \justifies
    {\mathcal{S}_{(q,Z)}:=\mathcal{S}_{(q,Z)}\cup\{(q'',Z')\}\quad \mbox{unless}\ \exists(q'',Z'')\in\mathcal{S}_{(q,Z)}\;Z'\preceq_{q''}Z''}
    \using{[\mathsf{local}]}
  \end{prooftree}
\]

\[
  \begin{prooftree}
    {(q,Z)\in\mathfrak{S}\quad (q',Z')\in\mathcal{S}_{q,Z}\quad q'\xrightarrow{x\in I}q''\quad Z''=Test(Z',x\in I)}
    \justifies
    {\mathcal{S}_{(q,Z)}:=\mathcal{S}_{(q,Z)}\cup\{(q'',Z'')\}\quad \mbox{unless}\ \exists(q'',Z''')\in\mathcal{S}_{(q,Z)}\;Z''\preceq_{q''}Z'''}
    \using{[\mathsf{test}]}
  \end{prooftree}
\]

\[
  \begin{prooftree}
    {(q,Z)\in\mathfrak{S}\quad (q',Z')\in\mathcal{S}_{q,Z}\quad q'\xrightarrow{x\leftarrow I}q''\quad Z''=Set(Z',x\leftarrow I)}
    \justifies
    {\mathcal{S}_{(q,Z)}:=\mathcal{S}_{(q,Z)}\cup\{(q'',Z'')\}\quad \mbox{unless}\ \exists(q'',Z''')\in\mathcal{S}_{(q,Z)}\;Z''\preceq_{q''}Z'''}
    \using{[\mathsf{assign}]} 
  \end{prooftree}
\]

\[
  \begin{prooftree}
    {(q(\mathcal{A}_i),Z)\in\mathfrak{S}\quad (q',Z')\in\mathcal{S}_{q(A_i),Z}\quad \mathcal{A}_i\xrightarrow{\mathsf{push}}\mathcal{A}_j
    \quad Z''=Reset(Z',X_c\cup\{\vdash^\bullet\})}
    \justifies
    {\mathfrak{S}:=\mathfrak{S}\cup\{(q_0(\mathcal{A}_j),Z'')\}\quad \mathcal{S}_{(q_0(\mathcal{A}_j),Z'')}\quad
    \mbox{unless}\ \exists (q_0,\mathcal{A}_j,Z''')\in\mathfrak{S}\;Z''\sim_{q_0(A_j)}Z'''}
    \using{[\mathsf{pop}]}
  \end{prooftree}
  \mbox{where $q_0$ is the initial location of $\mathcal{A}_j$}.
\]

\[
  \begin{prooftree}
    {\begin{array}{l}
      (q(\mathcal{A}_i),Z)\in\mathfrak{S}\quad (q',Z')\in\mathcal{S}_{(q(\mathcal{A}_i),Z)}\quad
      \mathcal{A}_i\xrightarrow{\mathsf{push}}\mathcal{A}_j\quad Z''\sim_{q_0(\mathcal{A}_j)}Z_1\quad
      Z''=Reset(Z',X_c\cup\{\vdash^\bullet\})\\
      (q_0(\mathcal{A}_j),Z_1)\in\mathfrak{S}\quad (q'_1,Z'_1)\in\mathcal{S}_{(q_0(\mathcal{A}_j))}
      \quad q'_1\in F(\mathcal{A}_j)\quad \mathcal{A}_j\mathcal{A}_i(q')\xrightarrow{\mathsf{pop}}\mathcal{A}_i(q_2)\quad
      Z_2=Up(Z'\odot Z'_1)
    \end{array}}
    \justifies
    {
    \mathcal{S}_{(q(\mathcal{A}_i),Z)}:=\mathcal{S}_{(q(\mathcal{A}_i),Z)}\cup\{(q_2,Z_2)\}\quad
    \mbox{unless}\ \exists(q_2,Z'_2)\in\mathcal{S}_{(q(\mathcal{A}_i),Z)}\;Z_2\preceq_{q_2}Z'_2
    }
    \using{[\mathsf{pop}]}
  \end{prooftree}
\]
\end{document}