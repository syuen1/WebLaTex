\documentclass{llncs}
\usepackage{amsmath,amsfonts,amssymb}
\include{prooftree}
\newtheorem{Def}{Definition}
\newtheorem{Lem}{Lemma}
\newcommand{\nnegreal}{\mathbb{R}^{\geq 0}}
\newcommand{\ulim}[1]{\mathsf{ub}(#1)}
\newcommand{\llim}[1]{\mathsf{lb}(#1)}
\newcommand{\zsub}[1]{(\!\vert{#1}\vert\!)}
\newcommand{\pushact}{\mathsf{push}}
\newcommand{\popact}{\mathsf{pop}}

\title{A Zone-based Reachability Analysis for Nested Timed Automata}
\author{Seiichirou Tachi\inst{1} \and Shoji Yuen\inst{1} \and Mizuhito Ogawa\inst{2}}
\institute{Graduate School of Informatics, Nagoya University, Japan \and
Japan Advanced Institute of Science and Technology< Japan}
\begin{document}
\maketitle

\section{Introduction}

\section{Timed Automata}

$\nnegreal$ denotes the set of non-negative real numbers. 
Given a set of clocks $X$ an {\em atomic clock guard} is in the form of
either $x\triangleleft c$ or $x-y\triangleleft c$ where $x,y\in X$, 
$\triangleleft\in\{<,\leq\}$ and $c\in\mathbb{N}$.  A {\em clock guard} is a conjunction
of atomic clock guards, and we write $\mathcal{B}(X)$ for the set of clock guards.
A set of clock valuations for clocks $X$ is written as $V_X:X\rightarrow\nnegreal$.

%  An {\em interval}
% is one of $[\ell,u]$, $(\ell,u]$, $[\ell,u')$, and $(\ell, u')$ where
% $\ell,u\in\nnegreal$ and $u'\in\nnegreal\cup\{\infty\}$.  For an
% interval, `$[$' and `$]$' are the {\em closed} lower bound and upper bound
% respectively and `$($' and `$)$' are the {\em open} lower bound and upper bound.
% The open upper bound $\infty$ shows the interval has no upper bound.  The
% set of all intervals is $\mathcal{I}$ For an interval $I$, $\llim{I}=\ell$,
% $\ulim{I}=u'$ or $\ulim{I}=u$ for
% $[\ell,u]$, $(\ell,u)$, $[\ell,u')$, and $(\ell,u')$.

% \begin{Def}[Timed Automata]
%   A timed automaton is a tuple $\mathcal{A}=(Q,q_0,X,\Delta)$ where
% $\nnegreal$ denotes the set of non-negative real numbers and 
% $\mathbb{Z}$ denotes the set of integers.
% Given a set of clocks $X$, an atomic time constraint over $X$ is
% $x-y\triangleleft c$ where $x,y\in X\cup\{0\}$, $\triangleleft\in\{<,\leq\}$, and $c\in\mathbb{Z}$.
% A time constraint over $X$ is a set of atomic time constraints over $X$, and the set of time
% constraints is written as $\mathcal{B}(X)$.  A time constraint is understood as the conjunction
% of its members.

% An {\em interval}
% is one of $[\ell,u]$, $(\ell,u]$, $[\ell,u')$, and $(\ell, u')$ where
% $\ell,u\in\nnegreal$ and $u'\in\nnegreal\cup\{\infty\}$.  For an
% interval, `$[$' and `$]$' are the {\em closed} lower bound and upper bound
% respectively and `$($' and `$)$' are the {\em open} lower bound and upper bound.
% The open upper bound $\infty$ shows the interval has no upper bound.  The
% set of all intervals is $\mathcal{I}$ For an interval $I$, $\llim{I}=\ell$,
% $\ulim{I}=u'$ or $\ulim{I}=u$ for
% $[\ell,u]$, $(\ell,u)$, $[\ell,u')$, and $(\ell,u')$.

\begin{Def}[Timed Automata]
  A timed automaton is a tuple $\mathcal{A}=(Q,q_0,X,\Delta)$ where
  \begin{itemize}
    \item $Q$ is a finite set of control locations 
    with the initial location $q_0\in Q$,
    %\item $F\subseteq Q$ is the set of final locations,
    \item $X$ is a finite set of clocks,
    % \item $\Delta\subseteq Q\times\mathcal{O}\times Q$ where $\mathcal{O}$ is a set of operations.
    % A transition $\delta\in\Delta$ is a triplet $(q_1,\phi,q_2)$ written as $q_1\xrightarrow{\phi}q_2$
    % with $\phi\in\mathcal{O}$ as one of the following operations over $X$:
    % \begin{list}{}{}
    %   \item [{\bf Local}] $\epsilon$, an internal operation;
    %   \item [{\bf Test}] $x\in I?$ where $x$ is a clock and $I\in\mathcal{I}$; and
    %   \item [{\bf Assignment}] $x\leftarrow I$ where $x$ is a clock and $I\in\mathcal{I}$.
    % \end{list}
    \item $\Delta\subseteq Q\times\mathcal{B}(X)\times 2^X\times Q$,
    \item $F\subseteq Q$ is the set of accepting locations.
  \end{itemize}
\end{Def}

% We write $\mathcal{A}(X)$ to explicitly show that $\mathcal{A}$'s clocks are $X$.

A clock valuation over $X$ is a map from $X$ to $\nnegreal$.   The set of clock valuations over $X$
is written as $N_X$.  The clock valuation assigning 0 to $X$ is denoted by $0_X$.

% \begin{Def}[Semantics of Timed automata]
%   Given a timed automaton $\mathcal{A}=(Q,q_0,X,\Delta)$, a configuration
%   of $\mathcal{A}$ is $(q,\nu)$ where $q\in Q$ and $\nu\in N_X$.
%   A transition $(q_1,\nu_1)\xrightarrow{\phi}(q_2,\nu_2)$ where
%   \begin{itemize}
%     \item $(q_1,\nu_1)\xrightarrow{\epsilon}(q_2,\nu_2)$ with $\nu_1=\nu_2$ and $(q_1,\epsilon,q_2)\in\Delta$;
%     \item $(q_1,\nu_1)\xrightarrow{x\in I?}(q_2,\nu_2)$ with $(q_1,x\in I?,q_2)\in\Delta$, $\nu_1=\nu_2$ and $\nu_1(x)\in I$; and
%     \item $(q_1,\nu_1)\xrightarrow{x\leftarrow I}(q_2,\nu_2)$ with $(q_1,x\leftarrow I,q_2)\in\Delta$, $\nu_2=\nu_1[x\mapsto r]$ where $r\in I$
%   \end{itemize}
% is written as $V_X$.  The clock valuation assigning 0 to $X$ is denoted by $0_X$.
% For a time constraint $g$, we write $\nu\models g$ when a clock valuation $\nu$ satisfies all atomic
% time constraint in $g$.  For $R\subseteq X$, $[R]\nu$ gives a clock valuation defined as follows:
% \[
% [R]\nu(x)=\begin{cases}0 & \mbox{if $x\in R$}\\\nu(x) & \mbox{otherwise}\end{cases}  
% \]

\begin{Def}[Semantics of Timed automata]
  Given a timed automaton $\mathcal{A}=(Q,q_0,X,\Delta,F)$, a configuration
  of $\mathcal{A}$ is $(q,\nu)$ where $q\in Q$ and $\nu\in V_X$.
  A transition $(q_1,\nu_1)\xrightarrow{t}(q_2,\nu_2)$ where $t$ is either $d\in\nnegreal$ or
  $\varepsilon$ with:
  \begin{itemize}
    \item $(q_1,\nu_1)\xrightarrow{d}(q_1,\nu_1+d)$;
    \item $(q_1,\nu_1)\xrightarrow{g,R}(q_2,[R](\nu_1))$ where $(q_1,g,R,q_2)\in\Delta$ and $\nu_1\models g$.
  \end{itemize}
\end{Def}

\section{Nested Timed Automata}

\begin{Def}
  Given a disjoint pair of clocks $X_g$ and $X_\ell$, let a set of timed automata
  be $T=\{\mathcal{A}_0,\cdots,\mathcal{A}_n\}$ where $\mathcal{A}_i$ is a timed
  automaton $(Q_i,q^0_i,X_\ell\cup X_g,\Delta_i)$.  We assume $Q_i\cap Q_j=\varnothing$ if $i\not=j$
  and we write $Q=\cup_i Q_i$ and $Q^0=\cup_{i}\{q^i_0\}$.
  A {\em nested timed automaton} (NeTA) is given by $\mathcal{N}=(T,\mathcal{A}_0,\Delta,X_g)$ where
  \begin{itemize}
    \item $A_0\in T$ is the initial timed automaton,
    \item $X_g$ is the set of global clocks, and
    \item $\Delta_g\subseteq (Q\times\{\pushact\}\times Q^0)\cup(Q\times Q\times\{\popact\}\times Q)$
  \end{itemize}
\end{Def}

For simplicity, every $A_i$ has the same set of local clocks $X_\ell$. 

Note that since $Q_i$ is disjoint to each other, each $\pushact$ rule specifies
the pushed automaton $A_i$ with $q\in Q(A_i)$.  Similarly, each $\popact$ rule specifies
a pair of automata $A_i$ and $A_j$ and $A_j$ is popped.  To explicitly show which
automaton is involved, we write $q(\mathcal{A}_i)$ when $q\in Q(\mathcal{A}_i)$. 

% \begin{list}{}{}
%   \item[$\pushact$] $A_i(q)\xrightarrow{\pushact}A_j$ for $(q,\pushact,q^0_j)\in\Delta_g$, and
%   \item[$\popact$] $\mathcal{A}_i(q_2)\mathcal{A}_j(q_1)\xrightarrow{\popact}\mathcal{A}_i(q_3)$ 
%   for $(q_2q_1,\popact,q_3)\in\Delta_g$ with 
%   $q_1\in Q(\mathcal{A}_j),q_2,q_3\in Q(\mathcal{A_i})$.
% \end{list}

\begin{Def}
  A {\em configuration} of NeTA is given by $(c,\mu)$ where $c\in\mathcal{C}^\ast$ with 
  $\mathcal{C}=\bigcup_i(Q(\mathcal{A}_i)\times V_{X_\ell})$ and $\mu\in V_{X_g}$.

  \begin{itemize}
    \item $(c_1,\mu_1)\xrightarrow{t}(c_2,\mu_2)$ where $t\in\mathbb{R}_{\geq 0}$, $c_2=c_1+t$ and $\mu_2=\mu_1+t$;
    \item $(c(q_1,\nu_1),\mu_1)\xrightarrow{g,R}(c(q_2,\nu_2),\mu_2)$ if 
    $(q_1,\nu_1\cup \mu_1)\xrightarrow{g,R}(q_2,\nu_1\cup \mu_2)\in\Delta(\mathcal{A}_i)$;
    \item $(c(q,\nu),\mu)\xrightarrow{\pushact}(c(q,\nu)(q_0(\mathcal{A}_i),0_{X_\ell}),\mu)$
    if $q\xrightarrow{\pushact}q_0(\mathcal{A}_i)\in\Delta_g$; and
    \item $(c(q_1,\nu_1)(q_2,\nu_2),\mu)\xrightarrow{\popact}(q_3,\nu_1),\mu)$
    if $q_1q_2\xrightarrow{\popact}q_3\in\Delta_g$
  \end{itemize}

\noindent
  where $c+t$ for $(c+t)[i]=(q^i,\nu^i+t)$ with $c[i]=(q^i,\nu^i)$ where $c[i]$ is the $i$-th element in $c$
  for $1\leq i\leq |c|$.
\end{Def}

The {\em reachability problem} of NeTA is to check if there is a sequence of transitions from 
$((q_0(\mathcal{A}_0),0_{X_\ell}),0_{X_g})$ to $(c(q,\nu),\mu)$ for some $\nu$ and $\mu$
given $\mathcal{N}$ and $q\in Q_\mathcal{N}$.

\section{Push-down automata over Zones}

Given a set of clocks $X_g$ and $X_\ell$, the set of items $Y$ derived from $X_g$ and $X_\ell$ is 
$\{0,\vdash,\vdash^\bullet\}\cup\{x^\bullet|x\in X_g\}\cup X_\ell\cup X_g$ and $Y_{clk}$ for 
$Y\{0\}$.

For a set of items $Y$,
A {\em zone} over $Y$ is $Z\subseteq Y\times Y\times \{\leq,<\}\times(\mathbb{Z}\cup\{\infty\})$
satisfying the following conditions.
\begin{itemize}
  \item $y\not=y'$ for $(y,y',\preceq,c)\in Z$; and
  \item $(y_1,y'_1,\preceq_1,c_1)\in Z$ implies $y_1\not=y_2$ or $y_2\not=y'_2$ for 
  all $(y_2,y'_2,\preceq_2,c_2)\in Z\backslash(y_1,y_1',\preceq_1,c_1)$
\end{itemize}

For $(y,y',\preceq,c)\in Z$, we write $y-y'\preceq c$ where $\preceq\in\{<,\leq\}$.
By the second condition, it is ensured that the pair $(y,y')$ is unique in a zone.
Thus, a zone can be described as a form of the {\em difference bound matrix} where
$(\preceq, c)$ is placed in the column labelled by $y'$ and the row labelled by $y$
for $y-y'\preceq c$.


%For clocks $X$, the PDZ-items $Z_x=X\cup X^\bullet\cup\{\vdash,\vdash^\bullet



\begin{itemize}
  \item $\mathsf{Test}(Z,x\in I)=Z\wedge\{x-\mathbf{0}\triangleleft\ulim{I},\mathbf{0}-x\triangleleft-\llim{I}\}$;
  \item $\mathsf{Free}(Z,Y)=(Z\ominus Y)\oplus Y$;
%  \item $\mathsf{Set}(Z,x\leftarrow I)=\mathsf{Free}(Z,\{x\})\wedge\{x-\mathbf{0}\triangleleft\ulim{I},\mathbf{0}-x\triangleleft-\llim{I}\}$;
  \item $\mathsf{Reset}(Z,Y)=\mathsf{Free}(Z,Y)\wedge\{y-\mathbf{0}\leq 0\;|\;y\in Y\}$;
  \item $\mathsf{Copy}(Z,x\leftarrow y)=\mathsf{Free}(Z,\{x\})\wedge\{x-y\leq 0,y-x\leq 0\}$;
  \item $Z\zsub{y\mapsto z}=\mathsf{Copy}(Z,z\leftarrow y)\ominus\{y\}$
\end{itemize}

\section{Simulation}

\begin{Def}
  A binary relation $\preceq$ on 
  $Q\times V_{X_g}\times V_{X_\ell}\times V_{X^\bullet_g\cup\{\vdash^\bullet\}}$ is a
  {\em simulation} if, whenever $(q,\mu,\nu,\mu^\bullet)\preceq (q,\mu',\nu',\mu'^\bullet)$:
  \begin{itemize}
    \item $(q,\mu+t,\nu+t,\mu^\bullet+t)\preceq (q,\mu'+t,\nu'+t,\mu'^\bullet+t)$;
    \item $(q,\mu,\nu,\mu^\bullet)\xrightarrow{g,R}(q_1,\mu_1,\nu_1,\mu^\bullet_1)$ implies
     for some $(\mu'_1,\nu'_1,\mu'^\bullet_1)$
      $(q,\mu',\nu',\mu'^\bullet)\xrightarrow{g,R}(q_1,\mu'_1,\nu'_1,\mu'^\bullet_1)$ and
      $(q_1,\mu_1,\nu_1,\mu^\bullet_1)\preceq(q_1,\mu'_1,\nu'_1,\mu'^\bullet_1)$;
    \item $q\xrightarrow{\pushact}q'$ implies
      $(q',\mu,\nu_0,\bullet(\mu)\uplus[\vdash^\bullet\mapsto 0])
        \preceq(q',\mu',\nu_0,\bullet(\mu')\uplus[\vdash^\bullet\mapsto 0])$\\
      \quad where $\bullet(\mu)(x^\bullet)=\mu(x)$ for $x\in X_g$;
    \item 
    % $v(\vdash^\bullet)=v'(\vdash^\bullet)$, and
    $qq''\xrightarrow{\popact}q'$ implies
    for all $(\mu_1,\nu_1,\mu^\bullet_1)$ such that $\mu_1(X_g)+\mu^\bullet(\vdash^\bullet)=\mu^\bullet(X_g^\bullet)$
    there exists $(\mu'_1,\nu'_1,\mu'^\bullet_1)$ such that 
    $\mu'_1(X_g)+\mu'^\bullet(\vdash^\bullet)=\mu'^\bullet(X_g^\bullet)$,
    $(q'',\mu_1,\nu_1,\mu^\bullet)\preceq(q'',\mu'_1,\nu'_1,\mu'^\bullet_1)$,
    and $(q',\mu,\nu_1+\mu^\bullet(\vdash^\bullet),\mu^\bullet_1+\mu^\bullet(\vdash^\bullet))
    \preceq (q',\mu',\nu'_1+\mu'^\bullet(\vdash^\bullet),\mu'^\bullet_1+\mu'^\bullet(\vdash^\bullet))$
  \end{itemize}

  $(q,Z)\preceq(q,Z')$ if for all $(\mu,\nu,\mu^\bullet)\models Z$, there exists $(\mu',\nu',\mu'^\bullet)$ 
  such that $(\mu',\nu',\mu'^\bullet)\models Z'$ and $(q,\mu,\nu,\mu^\bullet)\preceq(q,\mu',\nu',\mu'^\bullet)$.

\end{Def}

If $q\not=q'$, there exists no relation between $(q,v)$ and $(q',v')$.   We write
$v\preceq_q v'$ if $(q,v)\preceq (q,v')$.   Similarly, we write $Z\preceq_q Z'$
for $(q,Z)\preceq (q,Z')$. 

\noindent [The following lemma is yet to be proved]
\begin{lemma}
% A binary relation $\preceq_q$ on $V_{X_g}\uplus V_{X_\ell}$
% is a {\em simulation} if
% \begin{itemize}
%   \item for each delay $t\in\mathbb{R}_{\leq 0}$, $(q,v+t)\preceq (q,v'+t)$;
%   \item $(q,v)\xrightarrow{g,R}(q_1,v)$ implies 
%   $(q,v')\xrightarrow{g,R}(q_1,v')$;
% \end{itemize}

for all $Z\preceq_q Z'$,
% \begin{itemize}
%   \item $q=q'$;
%   \item $(q,Z)\xrightarrow{g,R}(q_1,Z_1)$ implies for some $Z'_1$ $(q',Z')\xrightarrow{g,R}(q_1,Z'_1)$
%   and $(q_1,Z_1)\preceq (q_1,Z'_1)$;
%   \item $(q,Z)\xrightarrow{\pushact}(q_1,Z_1)$ implies for some $Z'_1$ $(q',Z')\xrightarrow{\pushact}(q_1,Z'_1)$
%   and $(q_1,Z_1)\preceq (q_1,Z'_1)$; and
%   \item $(q,Z)(q_2,Z_2)\xrightarrow{\popact}(q_1,Z_1)$ and $Z_1\not=\bot$ imply $(q',Z')(q_2,Z_2)\xrightarrow{\popact}(q'_1,Z'_1)$
%   and $(q_1,Z_1)\preceq (q'_1,Z'_1)$ and $q_1=q'_1$
% \end{itemize}

\begin{itemize}
  % \item $(q,Z)\xrightarrow{g,R}(q_1,Z_1)$ implies for some $Z'_1$ $(q,Z')\xrightarrow{g,R}(q_1,Z'_1)$
  % and $Z_1\preceq_{q_1}Z'_1$;
  % \item  $q\xrightarrow{\pushact}q_0\in\Delta_g$ implies $\mathsf{push}(Z_1)\preceq_{q_0}\mathsf{push}(Z'_1)$; and 
  \item $Z_2\odot Z\not=\bot$ and $qq''\xrightarrow{\popact}q'\in\Delta_g$
  imply $Z_2\odot Z\preceq_{q'}Z'_2\odot Z'$
  for some $Z_2'$ such that $Z_2\preceq_{q''} Z_2'$
\end{itemize}
\end{lemma}

\begin{lemma}
  ${\sqsubseteq_{LU}}_q$ is a simulation relation. 
\end{lemma}

\begin{lemma}
  ${\sqsubseteq_{LU}}_q\cap{\sqsubseteq_{LU}}_q^{-1}$ has a finite index.
\end{lemma}
% \noindent
% We write $v\preceq_q v'$ for $(q,v)\preceq (q,v')$.  Given a pair of zones $Z,Z'$,
% $Z\preceq_q Z'$ if for all $v\in Z$ there exists $v'\in Z'$ such that
% $v\preceq_q v'$.

% \begin{Lem}
%   \begin{itemize}
%     \item If $(q,Z)\xrightarrow{\phi}(q_1,Z_1)$ and $Z\preceq_q Z'$, then
%   for some $Z'_1$, $(q,Z')\xrightarrow{\phi}(q_1,Z'_1)$ and $Z_1\preceq Z'_1$; and
%     \item If $(q,Z)\xrightarrow{t}(q,Z_1)$ and $Z\preceq_q Z'$, then 
%   there exists $Z'_1$ such that $(q,Z')\xrightarrow{t}(q,Z'_1)$ and 
%   $Z_1\preceq Z'_1$
%   \end{itemize}
% \end{Lem}

% \[
% \begin{prooftree}
% {}\justifies
% {\mathfrak{S}:=\{(q_0(A_0),Z_0)\}, \mathcal{S}_{(q_0(\mathcal{A}_0),Z_0)}:=\{(q_0(\mathcal{A}_0),Z_0)\}}
% \using{[\mathsf{start}]}
% \end{prooftree}
% \]

\[
\begin{prooftree}
  {}\justifies
  {\mathfrak{S}:=\{(A_0,Z_0)\}, \mathcal{S}_{(\mathcal{A}_0,Z_0)}:=\{(q_0(\mathcal{A}_0),Z_0)\}}
  \using{[\mathsf{start}]}
  \end{prooftree}
\]

\[
  \begin{prooftree}
    {(\mathcal{A}_i,Z)\in\mathfrak{S}\quad (q',Z')\in\mathcal{S}_{(\mathcal{A}_i,Z)}\quad q'\xrightarrow{g,R}q''\quad
    Z''=[R]\mathsf{Test}(Z',g)}
    \justifies
    {\mathcal{S}_{(\mathcal{A}_i,Z)}:=\mathcal{S}_{(\mathcal{A}_i,Z)}\cup\{(q'',Z'')\}\quad \mbox{unless}\ \exists(q'',Z'')\in\mathcal{S}_{(\mathcal{A}_i,Z)}\;Z''\preceq_{q''}Z'''}
    \using{[\mathsf{local}]}
  \end{prooftree}
\]

\[
  \begin{prooftree}
    {\mathcal{A}_i,Z)\in\mathfrak{S}\quad (q',Z')\in\mathcal{S}_{(\mathcal{A}_i,Z)}\quad q'\xrightarrow{\pushact}q_0(\mathcal{A}_j)
    \quad Z''=Reset(Z',X_c\cup\{\vdash^\bullet\})}
    \justifies
    {\mathfrak{S}:=\mathfrak{S}\cup\{(\mathcal{A}_j,Z'')\}\quad \mathcal{S}_{(\mathcal{A}_j,Z'')}=\{(q_0(\mathcal{A}_j),Z'')\}\quad
    \mbox{unless}\ \exists (\mathcal{A}_j,Z''')\in\mathfrak{S}\;Z''\sim_{q_0(A_j)}Z'''}
    \using{[\pushact]}
  \end{prooftree}
\]

\[
  \begin{prooftree}
    {\begin{array}{l}
      (\mathcal{A}_i,Z)\in\mathfrak{S}\quad (q',Z')\in\mathcal{S}_{(\mathcal{A}_i,Z)}\quad
      q'\xrightarrow{\pushact}q_0(\mathcal{A}_j)\quad Z''\sim_{q_0(\mathcal{A}_j)}Z_1\quad
      Z''=Reset(Z',X_c\cup\{\vdash^\bullet\})\\
      (\mathcal{A}_j,Z_1)\in\mathfrak{S}\quad (q'_1,Z'_1)\in\mathcal{S}_{(\mathcal{A}_j,Z_1)}
      \quad q'q_1'\xrightarrow{\popact}q_2\quad
      Z_2=Up(Z'\odot Z'_1)
    \end{array}}
    \justifies
    {
    \mathcal{S}_{(\mathcal{A}_i,Z)}:=\mathcal{S}_{(\mathcal{A}_i,Z)}\cup\{(q_2,Z_2)\}\quad
    \mbox{unless}\ \exists(q_2,Z'_2)\in\mathcal{S}_{(\mathcal{A}_i,Z)}\;Z_2\preceq_{q_2}Z'_2
    }
    \using{[\popact]}
  \end{prooftree}
\]
\end{document}